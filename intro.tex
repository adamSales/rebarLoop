Randomized controlled trials (RCTs) are an increasingly common tool
across commercial and scientific domains.
Large internet companies such such as Microsoft and Amazon conduct and
analyze tens of thousands of RCTS---called
``A/B tests'' in this context---each year \citep{kohavi2017surprising}. This practice is also a
common among both  smaller start-up companies and even individuals
interested in improving their products or websites through iterative
development informed by the use of A/B testing \citep{ries2011lean}.
Of course, RCTs have long been a prominant feature in the
pharmaceutical industry, as well as in social and biomedical
sciences.

One case in point is that of  the ASSISTments TestBed, an A/B testing
program that runs within ASSISTments, a computer-based learning
platform used by over 50,000 students a year for homework and
classwork.
With the ASSISTments TestBed, an education
researcher can propose A/B tests to run within ASSISTments.
On one night's assigned homework, for a given topic such as ``Adding
Whole Numbers'', or
``Factoring Quadratics Equations,'' students are individually randomized
into one of two conditions, for instance, video- or text-based
instructional feedback to compare their effects on an outcome such as
homework completion. The anonymized data associated with this study,
consisting of several levels of granularity and a breadth of
covariates describing both historical pre-study and within-study
student interaction, is made available to the proposing researcher
along with some simple descriptive statistics and automated
analyses. The Testbed is currently host to over 100 such randomized
controlled trials running in ASSISTments.

RCTs are famously free of confounding bias.
In fact, a class of estimators, often called ``design-based''
\citet[e.g.][]{schochet2015statistical} or ``randomization based''
\citet[e.g.][]{rosenbaum:2002a} estimate treatment effects without assuming
any statistical model other than whatever is implied by the
experimental design itself.
Design-based statistical estimators are guarenteed to be unbiased.
Their associated inference---standard errors, hypothesis tests,
confidence intervals---also come with guarentees regarding accuracy.
In many cases, these apply regardless of the sample size or
the characteristics of the data's parent distribution.

While RCTs can reliably provide, say, accurate confidence intervals
for a treatment effect, they cannot guarentee precision.
The statistical precision of RCT-based estimates is inherently limited
by the RCT's sample size, which itself is typically subject to a
number of practical constraints.
For instance, in one typical ASSISTments TestBed A/B test, a total of
294 students were randomized between two conditions, leading to a
standard error of roughly four percentage points estimating the effect
on homework completion.
This standard error is too large to either determine the direction of
a treatment effect or rule out clinically meaningful effect sizes.

In contrast, observational data drawn from large administrative
datasets can frequently be brought to bear on some of the same
questions.
Analysis of observational data, unlike RCTs, requires a number of
untestable modeling assumptions, chief among them the assumption of no
unmeasured confounding.
Consequently, treatment effect estimates from observational data
cannot boast the same guarentees to accuracy as estimates from RCTs.
That said, in many cases they boast a much larger sample---and, hence,
greater precision---than equivalent RCTs.

In many cases, observational and RCT data coexist within the very same
database.
For instance, covariate and outcome data from a biomedical RCT may be
drawn from an administrative dataset that contains equivalent data for
patients who did not participate in the study and were not randomized
between conditions.
Similarly, data from the ASSISTments TestBed lives alongside
equivalent data from other ASSISTments users who, say, were not
assigned to the relvant homework topic when the experiment was
underway.
We refer to these study non-participants as the ``remnant'' from the
study \citep[c.f.][]{rebarPaper}.

Along similar lines, administrative datasets in biomedical,
computer-based, and other realms often feature increadibly rich and
high dimensional covariate data.
In the ASSISTments TestBed context, a given
trial is likely to consist of just one night’s homework, but the
student may have been previously assigned 100 homeworks in the school
year, each of which comes with a granular dataset including any number
of potentially relevant performance metrics.
Rich covariate data can dramatically increase the precision of
experimental estimates.
However, exploiting high-dimensional data requires algorithms or
models whose quality is itself often a function of sample size---large
samples may be necessary for researchers to most effectively use
covariates to improve statistical precision.

This paper will present a novel method, called ReLOOP, to estimate
treatment effects
from  RCT data while incorporating high-dimensional covariate data,
large observational remnant data, and cutting-edge machine learning
prediction algorithms to improve precision---sometimes dramatically.
It does so without compromising the accuracy guarenees of traditional
design-based RCT estimators, yielding unbiased point estimates and
sampling variance estimates that are conservative in expectation; in
fact, ReLOOP is itself design-based.
In particular, ReLOOP prevents ``bias leakage'': bias that might have
occurred due to differences between the remnant and the experimental
sample, biased or incorrect modeling of covariates, or other data
analysis flaws, does not leak into the ReLOOP estimator.
ReLOOP combines two recent causal methods: rebar \citep{rebarEDM,jreeTestBed},
which incorporates high dimensional remnant data into RCT estimators,
and LOOP \citep{loop}, which uses machine learning algorithms for
covariate adjustment in RCTs.
The paper will focus on the challenge of precisely estimating
treatment effects from a published set of 22 TestBed experiments \citep{data},
using prior log data from experimental participants and
non-participants in the ASSISTments system.

The nexus of machine learning and causal inference has recently experienced
rapid and exciting development.
This has included novel methods to analyze observational studies
\citep[e.g.][]{diamond2013genetic},
to estimate subgroup effects \citep[e.g.][]{bothways,kunzel2018transfer}, or to optimally
allocate treatment (e.g. \cite{uplift}, \cite{siyuan}).
Other developments share our goal, i.e. improving the precision of
average treatment effect estimates from RCTs.
These include \citet{bloniarz2016lasso}, which uses the Lasso regression
estimator \citep{tibshirani1996regression} to analyze experiments
\citep[also see][]{belloni2014inference}, and the Targeted Learning
\citep{van2011targeted} framework,
which combines ensemble machine-learning with semiparametric maximum
likelihood estimation.
Few of these methods are design based, and to our knowledge, no extant
method uses data from the remnant to estimate effects.

The next section will review cruicial background material: design
based RCT analysis and covariate adjustment, rebar, and LOOP.
Section \ref{sec:theMethod} will present our main methodological
contribution, ReLOOP.
Section \ref{sec:simulations} will demonstrate its potential with a
simulation study.
Section \ref{sec:assistments} will use ReLOOP to estimate treatment
effects in the TestBed experiments, and Section \ref{sec:conclusion}
will conclude.